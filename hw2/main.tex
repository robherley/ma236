\documentclass[11pt]{article}
\usepackage{geometry}
\usepackage{amsmath}
\usepackage{xcolor}

\newcommand{\red}[1]{\textcolor{red}{#1}}

\begin{document}
\title{MA-236: Homework 2}
\author{Rob Herley}
\maketitle

\begin{center}
I pledge my honor that I have abided by the Stevens Honor System.
\end{center}

\section*{1.18 Problems}
\subsection*{(3) Show that the following argument is valid:}
$$A \lor B, A \rightarrow C, B \rightarrow C, \text{so} \ C$$

\noindent
To compute this efficiently, we find try to find a counter example such that all
the premises are true and the conclusion is false. We can cut the table down to
premises we know that can possibly produce counter examples:
\begin{center}
\begin{tabular}{ccc|ccc|c}
$A$ & $B$ & $C$ & $A \lor B$ & $A \rightarrow C$ & $B \rightarrow C$ & $C$ \\ \hline
t & t & f &   \red{t}  & f       &  f  & \red{f} \\
t & f & f &   \red{t}  & f       &  t  & \red{f} \\
f & t & f &   \red{t}  & \red{t} &  f  & \red{f} \\
f & f & f &   f        &   t     &  t  & \red{f}
\end{tabular}
\end{center}

\noindent
Full computation:
\begin{center}
\begin{tabular}{ccc|ccc|c}
$A$ & $B$ & $C$ & $A \lor B$ & $A \rightarrow C$ & $B \rightarrow C$ & $C$ \\ \hline
t & t & t & t & t & t & t \\
t & t & f & t & f & f & f \\
t & f & t & t & t & t & t \\
t & f & f & t & f & t & f \\
f & t & t & t & t & t & t \\
f & t & f & t & t & f & f \\
f & f & t & f & t & t & t \\
f & f & f & f & t & t & f
\end{tabular}
\end{center}

\noindent
Since we cannot find a counter example, the argument is valid.

\end{document}

